\documentclass{article}

\bibliographystyle{plain}

\usepackage{epsfig}
\usepackage{xcolor}
\usepackage{amsmath}

\input{abbrev.tex}

\begin{document}

\title{Communication test for EBCM framework}
\author{
    D. Devendran   \footnotemark[2] \and
    D. T. Graves    \footnotemark[1]
        }
\date{May 4, 2023 ce.}
\maketitle

\section{Introduction}

We are testing the communications infrastructure for {\tt
Chombo4/src/EBCM}, the bits of Chombo that handle calculations
within the context merged and
unmerged cut cell geometries.   

See Katz, et al (\cite{Katz 2023}) for a basic idea of the mathematical
framework in which higher order cut cell algorithms live.

\subsection{The blessed shortcut}

Important note: we are not dealing multivalued cells (cells in which
more than one computational volume lives in a Cartesian grid cell) for
EBCM calculations yet.  They are a difficult case that we can easily
avoid.  For EB aficionados, here are the technical reasons we are
taking this blessed shortcut:
\begin{itemize}
  \item Multivalued cells create enormous complications in data
    structures and require special graph-based data holders instead of
    much simpler box-based data holders.
 \item That means they require a whole new set of serialization
   routines that are difficult to debug.   
 \item Because we have cell merging, we are never forced to use
   multivalued cells.
 \item Chombo does not produces multivalued cells at the finest EB
   level, which is all the EBCM uses.
 \item Premature generalization is as bad as premature optimization.
\end{itemize}
Here is the downside of the blessed shortcut.
\begin{itemize}
\item Cell merging makes adaptivew mesh refinement (AMR)
  and geometric multigrid (GMG) more difficult.   
\item A volume's possible graph connections to finer and coarser
  volumes can no longer be inferred by its Cartesian cell.
\item The blessed shortcut makes the process of managing coarse and
  fine connections even more complicated because it forces merging at
  coarser levels. 
\item For the class of algorithms in which we are interested, however,
  both  reliable GMG and AMR are both still research  problems.
\item  Devendran, et al. provide some insight into the difficulties
  geometric multigrid in a higher order EB context \cite{Devendran2014}.
\end{itemize}

\section{The bit we are testing}





\section{Conclusions}



\footnotetext[1]{Lawrence Berkeley National Laboratory, Berkeley,
  CA. Research at LBNL was supported financially by the Office of
  Advanced Scientific Computing Research of the US Department of
  Energy under contract number DE-AC02-05CH11231.}
\footnotetext[2]{Ford Motor Company, Sunnyvale, CA.}
\renewcommand{\thefootnote}{\fnsymbol{footnote}}\
\bibliography{references}

\end{document}
