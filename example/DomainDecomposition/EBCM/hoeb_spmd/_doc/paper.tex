\documentclass{article}

\bibliographystyle{plain}

\usepackage{epsfig}
\usepackage{xcolor}
\usepackage{amsmath}

\input{abbrev.tex}

\begin{document}

\title{Communication test for EBCM framework}
\author{
    D. Devendran   \footnotemark[2] \and
    D. T. Graves    \footnotemark[1]
        }
\date{May 4, 2023 ce.}
\maketitle

\section{Introduction}

We are testing the communications infrastructure for {\tt
Chombo4/src/EBCM}, the bits of Chombo that handle calculations
within the context merged and
unmerged cut cell geometries.   

See Katz, et al (\cite{Katz2023}) for a basic idea of the mathematical
framework in which higher order cut cell algorithms live.

\subsection{The blessed shortcut}

Important note: we are not dealing multivalued cells (cells in which
more than one computational volume lives in a Cartesian grid cell) for
EBCM calculations yet.  They are a difficult case that we can easily
avoid.  For EB aficionados, here are the technical reasons we are
taking this blessed shortcut:
\begin{itemize}
  \item Multivalued cells create enormous complications in data
    structures and require special graph-based data holders instead of
    much simpler box-based data holders.
 \item That means they require a whole new set of serialization
   routines that are difficult to debug.   
 \item Because we have cell merging, we are never forced to use
   multivalued cells.
 \item In fact, cell merging makes even some of the
   regular grid redundant (which is a complication we already manage
   when it comes to covered cells).
 \item Chombo does not produces multivalued cells at the finest EB
   level (which is all the EBCM infrastructure uses).
 \item Premature generalization is bad and can lead to all kinds of
   problems.   
\end{itemize}
Here is the downside of the blessed shortcut: cell merging makes
geometric multigrid (GMG) more difficult.  With cell merging, managing
coarse and fine graph connections becomes more complicated. A volume's
possible graph connections to finer and coarser volumes can no longer
be inferred by its Cartesian cell. On the other hand, for the class of
algorithms in which we are interested, reliable GMG is an ongoing
research problem.  Devendran, et al. \cite{Devendran2014}
provide some insight into the difficulties geometric multigrid in a
higher order EB context.  The current effort will only use PETSc
\cite{petsc-user-ref, petsc-efficient} algebraic multigrid (AMG)
solvers.

\section{The bit we are testing}

We need to test two bits of infrastructure, both of them related to
SPMD communication.  
\begin{itemize}
  \item We need ghost cells for the geometric data which lives in {\tt
    EBCM::MetaDataLevel }.
  \item We need ghost cells for the simulation data which lives in {\tt
    EBCM::HostLevelData}.
\end{itemize}
In both cases, we do the obvious test.
For $T \in \{\mbox{MetaDataLevel, HostLevelData} \}$:
\begin{itemize}
  \item We make $T_{dom}$, where one box covers the whole domain, and
    $T_{dist}$, which covers the domain with many boxes.
  \item The test passes if $T_{dom} == T_{dist}$ everywhere in the domain.
\end{itemize}

\section{Unreasonable expectations}

When the infrastructure is working the code should simply return that
all tests have passed.


\footnotetext[1]{Lawrence Berkeley National Laboratory, Berkeley,
  CA. Research at LBNL was supported financially by the Office of
  Advanced Scientific Computing Research of the US Department of
  Energy under contract number DE-AC02-05CH11231.}
\footnotetext[2]{Ford Motor Company, Sunnyvale, CA.}
\renewcommand{\thefootnote}{\fnsymbol{footnote}}\
\bibliography{references}

\end{document}
