\documentclass{article}
\title{CPScale Instruction Manual}
\author{Dan Graves}
\begin{document}
\maketitle
\section{Introduction}

This docuement describes how to use CPScale, the Chombo/Proto scaling
infrastructure.    Given a template for an input file, a template for
a batch script and some command line inputs, CPScale produces a
directory  for a series of runs which comprise a weak scaling test.
CPScale also generates a script to submit the jobs.   Between each
scale, the grid refinement is doubled.   Both the MPI ranks and the
processor count are multiplied by eight.

For now, the pattern for number of nodes is fixed at 1, 8, 64....
nnnn
\section{Command Line Arguments}
  \begin{itemize}
\item \tt{--help} provides helpful information (mostly what these
  arguments mean).
\item \tt{--procs}  The minimum number of MPI process.  Say, for
  example this is 3.   Then the next run in the scaling test will have
  24 processes and so on.   This should be smaller than the number of
  CPUs per node on the target machine.
\item \tt{--input} name the input file template.   Examples are
  provided.   Relative path is okay to use here.  pp
\item \tt{--batch} name the batch file template.   Examples are
  provided.   Relative path is okay to use here.pp
\item \tt{--max\_proc}  maximum number of MPI processes to use here.
  This is a not-to-exceed number so the largest run can be smaller
  than this.
\item \tt{--executable\_name} Name of the executable you want run.
  Absolute paths are best here but if you must use a relative path,
  remember that this will be run from a directory two levels deeper
  than the one you are in.   So if you run this script in the same
  directory that you have your executable, prefix the name with
  "../..''.
\item \tt{--nx}  Resolution of smallest calculation.
\item \tt{--prefix} Name of the test (directory will have this in the name).  
\end{itemize}


\section{Usage pattern}

Say a user has an executable file named \tt{blah.exe} in her directory
"/home/awesomeuser/execfiles''.   She wants 2 processes per node up to
128 processes.   She has an input template called \tt{inputs} and a
batch template called \tt{batch}.   The wants the minimum resolution
in each directon to be 64.   The command would look like
\begin{verbatim}
<prompt>  ./configure.scaling --nx=64 --input=inputs --batch=batch \
            --executable_name="/home/awesomeuser/execfiles/main.exe" \
           --procs=2 --max_proc=128
\end{verbatim}

  
\end{document}
