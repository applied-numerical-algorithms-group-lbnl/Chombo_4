\documentclass{article}

%\include{macros}

\bibliographystyle{plain}

\usepackage{epsfig}
\usepackage{xcolor}
%%\usepackage{amssymb}
\usepackage{amsmath}

\input{abbrev.tex}

\begin{document}

\title{Elliptic operator stability and small cells on  finite volume grids}
\author{
   \and D. Devendran    \footnotemark[2]
   \and D. T. Graves    \footnotemark[1]
        }

\maketitle

\begin{abstract}

  Numerical stability for elliptic operators is well understood.  To
  be numerically stable, a numerical operator for an elliptic equation
  cannot have non-zero positive real eigenvalues.  We explore a family
  of high order finite volume elliptic operators. We attempt to
  define for this family of algorithms the boundaries of stability
  within its parameter space.
  
\end{abstract}


\section{Underlying mathematics}

Embedded boundary (EB) grids are formed when one passes an  surface
through a Cartesian mesh.    For sufficiently complex geomtries, these methods
are very attactive because grid generation is a solved problem
even for moving grids \cite{MillerTrebotich2012}.
In the current context, we form the cutting surface as the zero
surface of a function  of space $I(\xbold)$, $\xbold \in R^D$.
For smooth ($I$),  moments can be generated to any
accuracy \cite{Schwartz2015}.



Formally, the underlying description of space
is given by rectangular control volumes on a Cartesian mesh
$\Upsilon_\ibold = [(\ibold-\half {\ubold})h, (\ibold+\half
{\ubold})h], \ibold \in \bigzbold^D$, where $D$ is the dimensionality
of the problem, $h$ is the mesh spacing, and ${\ubold}$ is the vector
whose entries are all one (note we use bold font $\ubold = (u_1, \dots, u_d,
\dots, u_D)$ to indicate a vector quantity).
Given an irregular domain $\Omega$, we
obtain control volumes $V_\ibold = \Upsilon_\ibold \bigcap \Omega$ and
faces $A_{\ibold,d\pm} = A_{\ibold \pm \half \ebold_d}$ which are the
intersection of the boundary of $\partial V_\ibold$ with the
coordinate planes $\{{\xbold}:x_d = (i_d \pm \half)h \}$ ($\ebold_d$ is
the unit vector in the $d$ direction).  We also
define $A_{B,\ibold}$ to be the intersection of the boundary of the
irregular domain with the Cartesian control volume: $A_{B,\ibold}
= \partial \Omega \bigcap \Upsilon_\ibold$. 

We use the compact notation
\begin{align*}
(\xbold - \xbar)^\pbold &= \prod\limits^D_{d=1} (x_d - {\bar x}_d)^{p_d} \\
\pbold! &= \prod\limits^D_{d = 1} p_d!
\end{align*}
Given a $D$-dimensional region of space $\ibold$, 
and a $D$-dimensional integer vector
$\pbold$, we define $m_\ibold^\pbold(\xbar)$ to be the $\pbold^{th}$
moment of $\ibold$ to the point $\xbar$.
\begin{equation}
m_\ibold^\pbold(\xbar)  =  \int\limits_{\ibold} (\xbold - \xbar)^\pbold dV^D.
\label{eqn::volmoment}
\end{equation}
This breaks out into three different cases, volumes, coordinate faces
and irregular faces.
A coordinate face $f_{12}$  is the area where two
volumes ($V_1, V_2$) intersect.
A irregular face $f_V$  is area where the cutting surface
intersects the volume $V$.
The volume moment for a particular power $\pbold$  is
given by
\begin{equation}
  m^\pold_V = \int\limits_{\ibold} (\xbold - \xbar)^\pbold dV
  \label{eqn::volmom}
\end{equation}
The $\pbold^{th}$ coordinate face moment for the area of where volumes $V_1, V2$
intersect is given by
\begin{equation}
  m^\pold_{f_{12}} = \int\limits_{A(V_1\bigcap V_2)} (\xbold - \xbar)^\pbold dA.
  \label{eqn::facmom}
\end{equation}
The cutting surface of the boundary $G$ is given by  the zero
surface of the  implicit function ($I(\xbold)= 0$) used in grid generation.
The $\pbold^{th}$ irregular face moment is given by an integral over
the intersection of this cutting surface and the Euclidean volume:
\begin{equation}
  m^\pold_{V}} = \int\limits_{A(V \bigcap G)} (\xbold - \xbar)^\pbold dA.
  \label{eqn::irrmom}
\end{equation}
These moments are the natural product of the grid generation algorithm
in \cite{Schwartz2015}.

Given a sufficiently smooth function $\psi$, we can approximate $\psi$
in the neighborhood of $\xbar$ using a Taylor expansion to order $P_T$:
\begin{equation}
\psi(\xbold)  =  \sum\limits_{p < P_T} C^p (\xbar -\xbar)^p
\label{eqn::taylor}
\end{equation}
where $C^p$ this appropriate Taylor coefficient.  In three dimensions,
\begin{equation}
  C^p =\frac{1}{p!}
      \frac{\partial^{p_0}}{\partial x_0}
      \frac{\partial^{p_1}}{\partial x_1}
      \frac{\partial^{p_2}}{\partial x_2}  (\psi).
\end{equation}  
In finite volume  methods, we define grid data to be
averages over volumes.   So if we discretize the smooth function
$\psi$, the average over the volume $\ibold$ is given by
\begin{equation*}
 V_\ibold <\psi>_\ibold = \int\limits_{V_\ibold} \psi(\xbold) dV
\end{equation*}
If we insert the Taylor expansion \ref{eqn::taylor}, we get a discrete
approximation to the smooth function $\psi$ that is accurate to order $P_T$.
\begin{equation*}
  <\psi>_\ibold = \frac{1}{V_\ibold} \sum\limits_{p < P_T} C^p m^p
  \label{eqn::vol}.
\end{equation*}
This forms a local polynomial expansion of $\psi$ around the volume
$\ibold$ expressed in terms moments ($m$).
Modern, higher order embedded boundary
methods use this description directly to generate stencil coefficients
\cite{Overton2022a, Devendran2017, Schwartz2015, Katz2023}.

\section{Family of finite volume elliptic operators}

Devendran, et al. \cite{Devendran2017} present an algorithm that
solves Poisson's equation to fourth-order accuracy using an EB grid.
Generalizing somewhat  the framework they present, one can define an
elliptic operator of any accuracy.

At each volume $V$, we define the neighborhood $\neigh_v$
volumes within a radius $R_s$ number of cells from the volume.
Fix $\xbar$ to be the distance from the center of the target cell.
Let's say there are
$N_v$ volumes in the neighborhood and $F_D$ domain faces and $F_C$ cut
cell faces in the domain.  Following (\cite{Devendran2017}), we create
an equation for each volume and boundary condition within the
neighborhood to form an  overdetermined system of equation for the
Taylor coefficients in $V$.    Each volume gets an equation of the
form of equation \ref{eqn::vol}.    Faces at the boundary domain with
Neumann boundary conditions get an equation of the form 
by We For each volume, we have The equations for the volumes and the
boundary conditions form an overdetermined  system.



\section{Stability and Conditioning}

The matrix $A_{P_T, P_W}$ for a particular polynomial order $P_T$ and
weighting exponent  $P_W$ may be unstable or poorly conditioned.   We
measure both while varying $P_T, P_W$ in both two and three
dimensions.    All these calculations are done using SLEPc and PETc
\cite{petsc-user-ref, petsc-efficient, slepc}.




\section{Conclusions}



\footnotetext[1]{Lawrence Berkeley National Laboratory, Berkeley,
  CA. Research at LBNL was supported financially by the Office of
  Advanced Scientific Computing Research of the US Department of
  Energy under contract number DE-AC02-05CH11231.}
\footnotetext[2]{Ford Motor Company, Sunnyvale, CA.}
\renewcommand{\thefootnote}{\fnsymbol{footnote}}\
\bibliography{references}

\end{document}
