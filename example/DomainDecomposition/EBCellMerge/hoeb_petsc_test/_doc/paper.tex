\documentclass{article}

%\include{macros}

\bibliographystyle{plain}

\usepackage{epsfig}
\usepackage{xcolor}
%%\usepackage{amssymb}
\usepackage{amsmath}

\input{abbrev.tex}

\begin{document}

\title{Elliptic operator stability and small cells on  finite volume grids}
\author{
        O. Antepara    \footnotemark[1]
   \and D. Devendran    \footnotemark[2]
   \and D. T. Graves    \footnotemark[1]
   \and H. Johansen     \footnotemark[1]
   \and N. Overton-Katz \footnotemark[1]
        }

\maketitle

\begin{abstract}

  Numerical stability for elliptic operators is well understood.  To
  be numerically stable, a numerical operator for an elliptic equation
  cannot have non-zero positive real eigenvalues.  We explore a family
  of elliptic operators and attempt to define the boundaries of
  stability for this family of algorithms.
  
\end{abstract}


\section{Introduction}
Embedded boundary (EB) grids are formed when one passes an  surface
through a Cartesian mesh.    For sufficiently complex geomtries, these methods
are very attactive because grid generation is a solved problem
\cite{MillerTrebotich2012}.

  
Devendran, et al. \cite{Devendran2017} present an algorithm that
solves Poisson's equation to fourth-order accuracy using an EB grid.
Katz, et al. \cite{Katz2023} looked more closely at the mechanics of
this algorithm and discussed solvability for stencil coefficients for
a while varying dimensionality, weighting
function exponents, and polynomial order.   They present results for
uncut, standard EB, and merged grids.

We take this analysis to the entire numerical system.  We vary these
same algorithm parameters and analyze the eigenstructure of the

In the current context, we form the cutting surface as the zero
surface of a function  of space $I(\xbold)$, $\xbold \in R^D$.
For smooth ($I$),  moments can be generated to any
accuracy \cite{Schwartz2015}.

Formally, the underlying description of space
is given by rectangular control volumes on a Cartesian mesh
$\Upsilon_\ibold = [(\ibold-\half {\ubold})h, (\ibold+\half
{\ubold})h], \ibold \in \bigzbold^D$, where $D$ is the dimensionality
of the problem, $h$ is the mesh spacing, and ${\ubold}$ is the vector
whose entries are all one (note we use bold font $\ubold = (u_1, \dots, u_d,
\dots, u_D)$ to indicate a vector quantity).
Given an irregular domain $\Omega$, we
obtain control volumes $V_\ibold = \Upsilon_\ibold \bigcap \Omega$ and
faces $A_{\ibold,d\pm} = A_{\ibold \pm \half \ebold_d}$ which are the
intersection of the boundary of $\partial V_\ibold$ with the
coordinate planes $\{{\xbold}:x_d = (i_d \pm \half)h \}$ ($\ebold_d$ is
the unit vector in the $d$ direction).  We also
define $A_{B,\ibold}$ to be the intersection of the boundary of the
irregular domain with the Cartesian control volume: $A_{B,\ibold}
= \partial \Omega \bigcap \Upsilon_\ibold$. 

Throughout this paper, we use the following compact notation:
\begin{align*}
(\xbold - \xbar)^\pbold &= \prod\limits^D_{d=1} (x_d - {\bar x}_d)^{p_d} \\
\pbold! &= \prod\limits^D_{d = 1} p_d!
\end{align*}
Given a point in space $\xbar$, and a $D$-dimensional integer vector
$\pbold$, we define $m_v^\pbold(\xbar)$ to be the $\pbold^{th}$
moment of the volume $V$ relative to the point $\xbar$.
\begin{equation}
m_v^\pbold(\xbar)  =  \int\limits_{V} (\xbold - \xbar)^\pbold dV
\label{eqn::volmoment}
\end{equation}

 
Given a sufficiently smooth function $\psi$, we can approximate $\psi$
in the neighborhood of $\xbar$ using a Taylor expansion to order $P_T$:
\begin{equation}
\psi(\xbold)  =  \sum\limits_{p < P_T} C^p (\xbar -\xbar)^p
\label{eqn::taylor}
\end{equation}
where $C^p$ this appropriate Taylor coefficient.  In three dimensions,
\begin{equation}
  C^p =\frac{1}{p!}
      \frac{\partial^{p_0}}{\partial x_0}
      \frac{\partial^{p_1}}{\partial x_1}
      \frac{\partial^{p_2}}{\partial x_2}  (\psi).
\end{equation}  
In finite volume  methods, we define grid data to be
averages over volumes.   So if we discretize the smooth function
$\psi$, the average over the volume $\ibold$ is given by
\begin{equation*}
 V_\ibold <\psi>_\ibold = \int\limits_{V_\ibold} \psi(\xbold) dV
\end{equation*}
If we insert the Taylor expansion \ref{eqn::taylor}, we get a discrete
approximation to the smooth function $\psi$ that is accurate to order $P_T$.
\begin{equation*}
 V_\ibold <\psi>_\ibold = \sum\limits_{p < P_T} C^p m^p. 
\end{equation*}
This forms a local polynomial expansion of $\psi$ around the volume
$\ibold$ expressed in terms moments ($m$) which are the natural
products of grid generation.  Modern, higher order embedded boundary
methods use this description directly to generate stencils
\cite{Overton2022a, Devendran2017, Schwartz2015}.



\section{Results}

\section{Conclusions}



\footnotetext[1]{Lawrence Berkeley National Laboratory, Berkeley,
  CA. Research at LBNL was supported financially by the Office of
  Advanced Scientific Computing Research of the US Department of
  Energy under contract number DE-AC02-05CH11231.}
\footnotetext[2]{Ford Motor Company, Sunnyvale, CA.}
\renewcommand{\thefootnote}{\fnsymbol{footnote}}\
\bibliography{references}

\end{document}
