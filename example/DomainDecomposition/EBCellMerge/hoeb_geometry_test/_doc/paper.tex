\documentclass{article}

%\include{macros}

\bibliographystyle{plain}

\usepackage{epsfig}
\usepackage{xcolor}
%%\usepackage{amssymb}
\usepackage{amsmath}

\newcommand{\Abold}{{\bf A}}
\newcommand{\abold}{{\bf a}}
\newcommand{\bbold}{{\bf b}}
\newcommand{\cbold}{{\bf c}}
\newcommand{\dbold}{{\bf d}}
\newcommand{\ebold}{{\bf e}}
\newcommand{\fbold}{{\bf f}}
\newcommand{\Fbold}{{\bf F}}
\newcommand{\gbold}{{\bf g}}
\newcommand{\hbold}{{\bf h}}
\newcommand{\ibold}{{\bf i}}
\newcommand{\jbold}{{\bf j}}
\newcommand{\kbold}{{\bf k}}
\newcommand{\lbold}{{\bf l}}
\newcommand{\mbold}{{\bf m}}
\newcommand{\Mbold}{{\bf M}}
\newcommand{\nbold}{{\bf n}}
\newcommand{\obold}{{\bf o}}
\newcommand{\pbold}{{\bf p}}
\newcommand{\qbold}{{\bf q}}
\newcommand{\rbold}{{\bf r}}
\newcommand{\sbold}{{\bf s}}
\newcommand{\tbold}{{\bf t}}
\newcommand{\ubold}{{\bf u}}
\newcommand{\vbold}{{\bf v}}
\newcommand{\wbold}{{\bf w}}
\newcommand{\xbold}{{\bf x}}
\newcommand{\ybold}{{\bf y}}
\newcommand{\zbold}{{\bf z}}
\newcommand{\bigzbold}{{\bf Z}}
\newcommand{\xspace}{\hspace{2 mm}}

\newcommand{\ahat}{{\hat a}}
\newcommand{\bhat}{{\hat b}}
\newcommand{\chat}{{\hat c}}
\newcommand{\dhat}{{\hat d}}
\newcommand{\ehat}{{\hat e}}
\newcommand{\fhat}{{\hat f}}
\newcommand{\ghat}{{\hat g}}
\newcommand{\hhat}{{\hat h}}
\newcommand{\ihat}{{\hat i}}
\newcommand{\jhat}{{\hat j}}
\newcommand{\khat}{{\hat k}}
\newcommand{\lhat}{{\hat l}}
\newcommand{\mhat}{{\hat m}}
\newcommand{\nhat}{{\hat n}}
\newcommand{\ohat}{{\hat o}}
\newcommand{\phat}{{\hat p}}
\newcommand{\qhat}{{\hat q}}
\newcommand{\rhat}{{\hat r}}
\newcommand{\shat}{{\hat s}}
\newcommand{\that}{{\hat t}}
\newcommand{\uhat}{{\hat u}}
\newcommand{\vhat}{{\hat v}}
\newcommand{\what}{{\hat w}}
\newcommand{\xhat}{{\hat x}}
\newcommand{\yhat}{{\hat y}}
\newcommand{\zhat}{{\hat z}}

\newcommand{\abar}{{\bar {\bf a}}}
\newcommand{\bbar}{{\bar {\bf b}}}
\newcommand{\cbar}{{\bar {\bf c}}}
\newcommand{\dbar}{{\bar {\bf d}}}
\newcommand{\ebar}{{\bar {\bf e}}}
\newcommand{\fbar}{{\bar {\bf f}}}
\newcommand{\gbar}{{\bar {\bf g}}}
%\newcommand{\hbar}{{\bar {\bf h}}}
\newcommand{\ibar}{{\bar {\bf i}}}
\newcommand{\jbar}{{\bar {\bf j}}}
\newcommand{\kbar}{{\bar {\bf k}}}
\newcommand{\lbar}{{\bar {\bf l}}}
\newcommand{\mbar}{{\bar {\bf m}}}
\newcommand{\nbar}{{\bar {\bf n}}}
\newcommand{\obar}{{\bar {\bf o}}}
\newcommand{\pbar}{{\bar {\bf p}}}
\newcommand{\qbar}{{\bar {\bf q}}}
\newcommand{\rbar}{{\bar {\bf r}}}
\newcommand{\sbar}{{\bar {\bf s}}}
\newcommand{\tbar}{{\bar {\bf t}}}
\newcommand{\ubar}{{\bar {\bf u}}}
\newcommand{\vbar}{{\bar {\bf v}}}
\newcommand{\wbar}{{\bar {\bf w}}}
\newcommand{\xbar}{{\bar {\bf x}}}
\newcommand{\ybar}{{\bar {\bf y}}}
\newcommand{\zbar}{{\bar {\bf z}}}

\newcommand{\dx}{{h}}
\newcommand{\nph}{{n + \frac{1}{2}}}
\newcommand{\iph}{{\ibold + \frac{1}{2}\ebold_d}}
\newcommand{\ipmh}{{\ibold \pm \frac{1}{2}\ebold_d}}
\newcommand{\imh}{{\ibold - \frac{1}{2}\ebold_d}}

\newcommand{\half}{\frac{1}{2}}
%\newcommand{\Fbold}{\mathbf{F}}
%\newcommand{\iphed}{{\ibold+\half\ebold^d}}
\newcommand{\deriv}{\partial}
\newcommand{\area}{\mathcal{A}}
\newcommand{\R}[1]{\mathbb{R}^{#1}}
\newcommand{\cf}{{\scriptscriptstyle F}}
\newcommand{\cn}{{\scriptscriptstyle N}}
\newcommand{\sEB}{{\scriptscriptstyle EB}}
\renewcommand{\vec}[1]{\mathbf{#1}} % Vector (bold)
\newcommand{\tens}[1]{\mathbf{#1}} % Tensor (bold)
\newcommand{\mat}[1]{\mathbf{#1}} % Matrix (bold)
\newcommand{\hatvec}[1]{\hat{\vec{#1}}} % Vector with a hat (bold)
\newcommand{\ddt}[1]{\frac{\partial #1}{\partial t}} % partial d/dt
\newcommand{\DDt}[1]{\frac{\text{d}#1}{\text{d}t}} % "total" d/dt
\newcommand{\ddr}[1]{\frac{\partial #1}{\partial r}} % d/dr
\newcommand{\ddx}[1]{\frac{\partial #1}{\partial x}} % d/dx
\newcommand{\ddy}[1]{\frac{\partial #1}{\partial y}} % d/dy
\newcommand{\ddz}[1]{\frac{\partial #1}{\partial y}} % d/dz
\newcommand{\ddxi}[1]{\frac{\partial #1}{\partial x_i}} % d/dx_i
\newcommand{\diverg}[1]{\nabla\cdot#1} % Divergence operator
\newcommand{\curl}[1]{\nabla \times #1} % Curl operator
\newcommand{\dOmega}{\text{d}\Omega} % Volume differential (Omega style)
\newcommand{\dV}{\text{d}V} % Volume differential
\newcommand{\dA}{\text{d}A} % Area differential
\newcommand{\labelEq}[1]{\label{eq:#1}\hbox{\tt #1 \quad}} % Use this to label an equation.
\newcommand{\refEq}[1]{(\ref{eq:#1})}   % Use this to reference an equation.
\newcommand{\labelSec}[1]{\label{sec:#1}} % Use this to label a section.
\newcommand{\refSec}[1]{\S\ref{sec:#1}} % Use this to reference a section.
\newcommand{\labelChap}[1]{\label{chap:#1}} % Use this to label a chapter.
\newcommand{\refChap}[1]{Chapter \ref{chap:#1}} % Use this to reference a chapter.
\newcommand{\labelApp}[1]{\label{app:#1}} % Use this to label an appendix.
\newcommand{\refApp}[1]{Appendix \ref{app:#1}} % Use this to reference an appendix.
\newcommand{\ib}{{\bf{i}}}    % bold italic i
\newcommand{\jb}{{\bf{j}}}    % bold italic j
\newcommand{\kb}{{\bf{k}}}    % bold italic k
\newcommand{\lb}{{\bf{l}}}    % bold italic l
\newcommand{\mb}{{\bf{m}}}    % bold italic m
\newcommand{\ub}{{\bf{u}}}    % bold italic u
\newcommand{\xb}{{\bf{x}}}    % bold italic x
\newcommand{\vb}{{\bf{v}}}    % (bold italic) VoF index
\newcommand{\fb}{{\bf{f}}}    % (bold italic) face index
\newcommand{\pb}{{\bf{p}}}    % (bold italic) multi-index p
\newcommand{\qb}{{\bf{q}}}    % (bold italic) multi-index q
\newcommand{\ebd}{{\bf{e}^d}}
\newcommand{\normal}[1]{\vec{n}_{#1}} % Outward normal.
\newcommand{\order}[1]{\mathcal{O}(#1)} % Order notation



\newcommand{\ed}{{\bf{e}_d}}
\newcommand{\Zbold}{{\bf{0}}}
\newcommand{\unitV}{\mathds{1}}
\newcommand{\eb}{\text{EB}}
\newcommand{\EB}{\text{EB}}
\newcommand{\vol}{\mathcal{V}}
\newcommand{\face}{\mathcal{F}}
\newcommand{\zerobold}{{\bf{0}}}
\newcommand{\xz}{{\bf{x}_0}}

\newcommand{\Dim}{D}
\newcommand{\dif}{\mathrm{d}}
\newcommand{\dt}{{\Delta t}}
\newcommand{\avg}[1]{\overline{#1}}
\newcommand{\avgI}[1]{\overline{#1}_{\ibold}}
\newcommand{\imhed}{{\ibold+\frac{1}{2}\ebold_d}}
\newcommand{\ipmhj}{{\ibold\pm\frac{1}{2}\ebold_d}}
\newcommand{\iphj}{{\ibold+\frac{1}{2}\ebold_d}}
\newcommand{\imhj}{{\ibold-\frac{1}{2}\ebold_d}}

\newcommand{\ipfed}[1]{\ibold+\frac{#1}{2}\ebold_d}
\newcommand{\imfed}[1]{\ibold-\frac{#1}{2}\ebold_d}
\newcommand{\ipmfed}[1]{\ibold\pm\frac{#1}{2}\ebold_d}
\newcommand{\eq}[1]{(\ref{#1})}

\newcommand{\nref}{n_{\text{ref}}}
\newcommand{\phio}{\phi^{\circ}}

\begin{document}

\title{Polynomial expansions of finite volume data in a cut cell context}
\author{P. Schwartz     \footnotemark[1]
   \and N. Overton-Katz \footnotemark[1]
   \and T. Ligocki      \footnotemark[1]
   \and H. Johansen     \footnotemark[1]
   \and D. T. Graves    \footnotemark[1]
   \and D. Devendran    \footnotemark[1]
   \and O. Anteparra    \footnotemark[1]}

\maketitle
\footnotetext[1]{Lawrence Berkeley National Laboratory, Berkeley,
  CA. Research at LBNL was supported financially by the Office of
  Advanced Scientific Computing Research of the US Department of
  Energy under contract number DE-AC02-05CH11231.}

\section
\begin{abstract}

  We discuss the mechanics of forming stable local polynomial
  expansions of finite volume data in the context of cut cell grids.
  We show that significant solvability issues can arise from certain weighting
  functions.    A very rudimentary but stable cell merger algorithm is
  presented in an attempt to show that these solvability issues are
  really an artifact of small-cell instabilties.
  
\end{abstract}
  
  function 
We present an algorithm to produce the necessary geometric information
for finite volume calculations in the context of Cartesian grids with
embedded boundaries.      Given an order of accuracy for the
overall calculation, we show what accuracy is required for each of the
geometric quantites and we demonstrate how to calculate the moments
using the divergence theorem.   We demonstrate that, for  a known
flux, these moments can be used to create a flux divergence of the
expected order.

\end{abstract}
\section{Introduction}



Throughout this paper, we use the following compact ``multi-index'' notation:
\begin{align*}
(\xbold - \xbar)^\pbold &= \prod\limits^D_{d=1} (x_d - {\bar x}_d)^{p_d} \\
\pbold! &= \prod\limits^D_{d = 1} p_d!
\end{align*}
Given a point in space $\xbar$, and a $D$-dimensional integer vector
$\pbold$, we define $m_v^\pbold(\xbar)$ to be the $\pbold^{th}$
moment of the volume $V$ relative to the point $\xbar$.
\begin{equation}
m_v^\pbold(\xbar)  =  \int\limits_{V} (\xbold - \xbar)^\pbold dV
\label{eqn::volmoment}
\end{equation}
Clearly, the volume of the cut cell $|V| = m_v^\zbold$,
  where $\zbold$ is the zero vector.
We define the face moments $m_{d\pm}^\pbold(\xbar)$ to be the $\pbold^{th}$  
moments (relative to the point $\xbar$) of the faces $A_{d\pm}$.
\begin{equation}
m_{d\pm}^\pbold(\xbar)  = \int\limits_{A_{d\pm}} (\xbold - \xbar)^\pbold dA
\label{eqn::facemoment}
\end{equation}
We define two moments corresponding to the embedded boundary face $A_B$: $m^{\pbold}_{B}$ and $m^{\pbold}_{B,d}$: 
\begin{equation}
m^{\pbold}_{B}(\xbar)  = \int\limits_{A_B} (\xbold
-\xbar)^\pbold \,  dA
\label{eqn::ebmomentB}
\end{equation}
and
\begin{equation}
m^{\pbold}_{B,d}(\xbar)  = \int\limits_{A_B} (\xbold
-\xbar)^\pbold \, n_d(\xbold) \, dA.
\label{eqn::ebmoment}
\end{equation}
Note that (\ref{eqn::ebmoment}) includes the normal to the embedded boundary face.
 
Given a sufficiently smooth function $\psi$, we can approximate $\psi$
in the neighborhood of $\xbar$ using a multi-dimensional Taylor expansion:
\begin{equation}
\psi(\xbold) = \sum_{|\qbold| < Q} \frac{1}{\qbold!}\psi^{(\qbold)}(\xbar)
\, (\xbold - \xbar)^\qbold + O(\dx^{Q}),
\label{eqn::taylor}
\end{equation}
with the multi-index partial derivative notation
\begin{equation}
\psi^{(\qbold)} = 
  \partial^{\qbold} \psi =
  \frac{\partial^{q_1}}{\partial x_1^{q_1}} \dots
  \frac{\partial^{q_D}}{\partial x_D^{q_D}} \psi \, .
\end{equation}
We express averages over volumes as 
\begin{equation*}
  \left< \nabla \cdot \Fbold \right>_V \equiv \frac{1}{|V|} 
  \int\limits_{V} \left( \nabla \cdot \Fbold \right) \, dV
\end{equation*}
We define the volume fraction $\kappa$ to be fraction of the volume
of the cell inside the solution domain, so that 
\begin{equation}
\kappa = h^{-D} |V| = h^{-D} m_v^\zbold \quad .
\label{eqn::kappa}
\end{equation}
